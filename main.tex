\documentclass{cdut_thesis}
\usepackage{listings} % 导入 listings 宏包,实现代码块
% 设置 listings 的全局样式
% 设置代码块样式
\lstset{
	language=TeX, % 语言设置为TeX
	basicstyle=\ttfamily, % 字体样式
	backgroundcolor=\color{gray!10}, % 背景色
	keywordstyle=\color{blue}, % 关键词颜色
	commentstyle=\color{green!60!black}, % 注释颜色
	showstringspaces=false, % 不显示字符串中的空格
	breaklines=true, % 自动断行
	numbers=left, % 行号位置
	numberstyle=\tiny\color{gray}, % 行号样式
	frame=single, % 代码块边框
	rulecolor=\color{black}, % 边框颜色
	xleftmargin=\parindent, % 左边距
}
\TitleCDUT{成都理工大学本科毕业论文非官方\LaTeX 模板}{Unofficial graduation thesis of Chengdu University of Technology \LaTeX template}
\KeywordCDUT{\LaTeX ;排版系统;毕业论文}{\LaTeX ;Typesetting system;Graduation thesis}% 注意用分号
\AuthorCDUT{姓名}
\StudentnumberCDUT{\quad}
\AdvisorCDUT{张三}{教授}
\MajorCDUT{数学与应用数学2班}
% 多少届的毕业生
\GradeCDUT{2024}

\begin{document}
%\includepdf{body/封面.pdf}% 不需要,若需要,将封面放到body中
% 摘要以及目录的页眉页脚
\fronthead
\makezhcover

\begin{chineseabstract}
%目的、方法、结果、意义
这是成都理工大学本科生毕业论文\textbf{非官方} \LaTeX 模板。
	
\textbf{v1.1:}本模板在\href{https://github.com/cs-whh/CDUT_thesis}{\color{red}GitHub:CDUT\_thesis项目}的基础上作了部分修改,以符合学校最新要求。
	
\textbf{注意:}本模板在编写过程中尽可能满足学校要求,由于原始规范主要针对Word。和\LaTeX 之间不可避免的差异加之编写者的水平有限,所以难免和学校提供的基于Word的样张存在细微差异,请谨慎使用!
\end{chineseabstract}



\begin{englishabstract}
This is an unofficial \ LaTeX template for graduate thesis of Chengdu University of Technology.
\end{englishabstract}


\newpage

\tableofcontents

\newpage
% 正文的页眉页脚
\texthead

%论文主体内容
\include{body/section_1}%第一章:绪论
\include{body/section_2}%第二章:相关理论
\include{body/section_3}%第三章:应用实践
\section{v1.1的重要修改内容:文献引用}
\textbf{2024年06月20日}:原版本的引用格式采用的是\textcolor{red}{顺序-编码制},与学校要求不符。本模板将引用格式修改为\textcolor{red}{著者-出版年制}。使用前置条件如下:
\subsection{制作或者生成bib文件}

\subsubsection{手动制作}
例如:
\begin{lstlisting}
@article{wanger2009,
	author    = {王二 and 张三 and 李四},
	key       = {wang2 er4 and zhang1 san1 & li3 si4},
	title     = {单引用测试,标题1},
	journal   = {journal},
	year      = {2010}	
	}
\end{lstlisting}
\begin{itemize}
	\item wanger2009:是文章的标签,引用时通过它,可以对应到文章。
	\item author:作者名称,不同作者按顺序用and分隔开。
	\item key:作者的拼音,数字表示声调。方便按拼音顺序排列参考文献
	\item title:文章标题
	\item journal:期刊名称
	\item year:出版年份
	...
\end{itemize}
不建议使用这种方式,挺麻烦的
\subsubsection{自动制作}
在阅读文献时使用zotero、endnote等文献管理器对文献进行管理,后期可以选中需要的文件一键导出参考文献的bib文件(如图\ref{参考文献导出示意图})。

\begin{figure}[h]
	\centering
	\includegraphics[width = 14cm]{figures/参考文献.png}
	\caption{zotero一键导出参考文献的bib文件}
	\label{参考文献导出示意图}
\end{figure}

\subsubsection{注意}
单独强调一下:无论是手动制作还是自动生成bib文件,只要是中文文献,就要手动加\textbf{key}值,以保证\textbf{中文文献在参考文献目录中能够按照拼音顺序排列}。例如:
\begin{lstlisting}
	@article{wanger2009,
		author    = {王二 and 张三 and 李四},
		key       = {wang2 er4 and zhang1 san1 & li3 si4},
	}
\end{lstlisting}
\subsection{两种引用格式}
有了bib文件,就可以在论文中插入引用了。假设最终需要引用的文献的bib文件如下:
\begin{lstlisting}
@article{wanger2009,
	author    = {王二 and 张三 and 李四},
	key       = {wang2 er4 and zhang1 san1 & li3 si4},
	title     = {单引用测试,标题1},
	journal   = {journal},
	year      = {2010}	
}
@article{zhangsan2010,
	author    = {张三 and 李四},
	key       = {zhang1 san1 & li3 si4},
	title     = {多引用测试,标题1},
	journal   = {journal},
	year      = {2010}	
}
@article{lisi2011,
	author    = {李四 and 张三},
	key       = {li3 si4 & zhang1 san1},
	title     = {多引用测试,标题2},
	journal   = {journal},
	year      = {2011}	
}
\end{lstlisting}
\subsubsection{第一种引用形式}
\textbf{示例1:}\cite{wanger2009}这是第一种引用形式。
\begin{lstlisting}
\cite{wanger2009}
\end{lstlisting}

\subsubsection{第二种引用形式}
\textbf{示例2:}这是第二种引用形式\citep{wanger2009}。
\begin{lstlisting}
\citep{wanger2009}
\end{lstlisting}

\textbf{示例3:}这是第二种引用形式\citep{wanger2009,zhangsan2010,lisi2011},引用多篇佐证本文观点。
\begin{lstlisting}
\citep{wanger2009,zhangsan2010,lisi2011}
\end{lstlisting}

写作时基本是用这两种引用形式,,英文文献与上面引用一致(会自动处理成英文版本)。

\textbf{示例4:}英文文献示例\citep{LiK2019}。
\begin{lstlisting}
\citep{LiK2019}
\end{lstlisting}


注意:只要bib文件是严格按照要求制作的,就不会在这里出现错误。所以最好学习使用文献管理器制作bib文件,而非手动制作。
\subsection{参考文献目录}

只要bib文件是严格正确的,参考文献目录会自动生成符合规定(中文在上,外文在下。文献名称按顺序排列下来等规则)的参考文献目录。所以这里不需要过多关注。

bib文件放在body文件夹中,其名称为:refer.bib。
\begin{lstlisting}
\phantomsection
\bibliography{body/refer.bib}
\end{lstlisting}

\section{v1.1的其他修改内容:一些格式上的调整}
\begin{itemize}
	\item 图、表、公式的caption采用"-"作为分隔符,比如:图1-1,表2-2,公式3-3,符合学校规定。
	\item 修改页眉内容:“成都理工大学20xx届学士学位论文(设计)”,符合学校规定
	\item 目录跳转问题:原模板在结论、致谢、参考文献等无法实现点击跳转,本模板解决了这个问题。
	\item 目录的其他问题:原模板引导点("......")过长,本模板作了调整,适合论文页码的位数$\le2$的论文;原模板目录会添加“摘要”、“目录”进目录,实际不需要。本模板作了删除处理。
	\item 页面距离的问题:原模版的文字到页眉距离过大,本模板将这段距离适当调小,看起来更为美观。由于页眉、页脚距离与word的原理不一样,所以这里面还存在着问题,需要后续进一步修改,以使得latex的排版结果与word一致。(其实影响不大,能解决最好)
	
\end{itemize}
%结论、致谢、参考文献
\include{body/conclusion}
\begin{thanks}
感谢老师感谢老师感谢老师感谢老师感谢老师感谢老师感谢老师感谢老师感谢老师。
\end{thanks}

% 引用 .bib 文件


\phantomsection
\bibliography{body/refer.bib}
%\include{body/appendix}%附录。一般可以没有
\end{document}

